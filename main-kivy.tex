\input{preamble}

\usepackage[none]{hyphenat}
%\iraggedright

\usepackage[margin=0.3in]{geometry}

\newcommand{\twotabvspace}{\vspace{0.2in}}

\begin{document}
\afterpage{%

    \clearpage% Flush earlier floats (otherwise order might not be correct)
    \thispagestyle{empty}% empty page style (?)
        \centering % Center table
        %\captionof{table}{Haber}% Add 'table' caption

        %\twotabvspace

\begin{tabular}{|p{2cm}|p{7.6cm}|p{6cm}|p{1cm}|}
\hline
\multicolumn{4}{|c|}{\textbf{Kivy: positioning and sizing}}\\
\hline\hline
\emph{Property}&\emph{Value}&\emph{For layouts}&\emph{For widgets}\\
\hline\hline

\code{size\_hint}
&A pair \code{w, h: w,} and \code{h} express a proportion (from 0 to 1 or \code{None}).
&Yes
&No\\\hline

\code{size\_hint\_x} \code{size\_hint\_y}
&A proportion from 0 to 1 or \code{None}, indicating width (\code{size\_hint\_x})
or height (\code{size\_hint\_y}).
&Yes
&No\\\hline

\code{pos\_hint}
&
Dictionary with one x-axis key (\code{x, center\_x,} or \code{right}) and one y-axis key (\code{y, center\_y,} or \code{top}). The values are proportions from 0
to 1.
&Yes
&No\\\hline

\code{size}
&A pair \code{w, h: w} and \code{h} indicating fixed width and height in pixels.
&Yes, but set \code{size\_hint: (None, None)}
&Yes\\\hline

\code{width}
&Fixed number of pixels.
&Yes, but set \code{size\_hint\_x: None}
&Yes\\\hline

\code{height}
&Fixed number of pixels.
&Yes, but set \code{size\_hint\_y: None}
&Yes\\\hline

\code{pos}
&A pair \code{x, y} indicating a fixed coordinate (\code{x, y}) in pixels.
&Yes, but don't use \code{pos\_hint}
&Yes\\\hline

\code{x, right} or \code{center\_x}
&Fixed number of pixels.
&Yes, but don't use \code{x, right} or \code{center\_x} in \code{pos\_hint}
&Yes\\\hline

\code{y, top} or \code{center\_y}
&Fixed number of pixels.
&Yes, but don't use \code{y, top} or \code{center\_y} in \code{pos\_hint}
&Yes\\\hline

\end{tabular}

\twotabvspace

\begin{tabular}{|p{2.5cm}|p{15cm}|}
\hline
\multicolumn{2}{|c|}{\textbf{Kivy: layouts}}\\
\hline\hline
\emph{Layout}&\emph{Details}\\
\hline\hline

\code{FloatLayout}
&
Organizes the widgets with proportional coordinates by the \code{size\_hint} and \code{pos\_hint} properties.
The values are numbers between 0 and 1, indicating a proportion to the window size.\\\hline

\code{RelativeLayout}
&
Operates in the same way that \code{FloatLayout} does, but the positioning properties (\code{pos, x, center\_x,
right, y, center\_y, top}) are relative to the \code{Layout} size and not the window size.\\\hline

\code{GridLayout}
&
Organizes widgets in a grid. You have to specify at least one of two properties – \code{cols} (for
columns) or \code{rows} (for rows).\\\hline

\code{BoxLayout}
&
Organizes widgets in one row or one column depending on whether the value of the \code{orientation}
property is \code{horizontal} or \code{vertical}.\\\hline

\code{StackLayout}
&
Similar to \code{BoxLayout}, but it goes to the next row or column when it runs out of space. There is
more flexibility to set the \code{orientation}. For example, \code{rl-bt} organizes the widgets in right-to-left,
bottom-to-top order. Any combination of \code{lr} (left to right), \code{rl} (right to left), \code{tb} (top to bottom), and
\code{bt} (bottom to top) is allowed.\\\hline

\code{ScatterLayout}
&
Works in a similar manner to \code{RelativeLayout} but allows multitouch gesturing for rotating, scaling,
and translating. It is slightly different in its implementation, so we will review it later on.\\\hline

\code{PageLayout}
&
Stacks widgets on top of each other, creating a multipage effect that allows flipping of pages using
side borders. Very often, we will use another layout to organize elements inside each of the pages,
which are simply widgets.\\\hline

\code{AnchorLayout}
&\\\hline

\end{tabular}

\twotabvspace

\begin{tabular}{|p{2.5cm}|p{15cm}|}
\hline
\multicolumn{2}{|c|}{\textbf{Kivy: layout advice}}\\
\hline
%&\\
\hline

Behavior quirks
&
\code{size\_hint, size\_hint\_x,} and \code{size\_hint\_y} work on all the layouts (except
\code{PageLayout}), but the behavior might be different. For example, \code{GridLayout} will
try to take an average of the \code{x} hints and \code{y} hints on the same row or column
respectively.\\\hline\hline

Size hint > 1.0
&
You should use values from 0 to 1 with \code{size\_hint, size\_hint\_x,} and
\code{size\_hint\_y}. However, you can use values higher than 1. Depending on the
layout, Kivy makes the widget bigger than the container or tries to recalculate a
proportion based on the sum of the hints on the same axis.\\\hline\hline

Positional hints
&
\code{pos\_hint} only works for \code{FloatLayout, RelativeLayout,} and \code{BoxLayout}. In
\code{BoxLayout}, only the axis-\code{x} keys (\code{x, center\_x, right}) work in the \code{vertical}
orientation and vice-versa for the \code{horizontal} orientation. An analogous rule
applies for the fixed positioning properties (\code{pos, x, center\_x, right, y,
center\_y,} and \code{top}).\\\hline\hline

Size hints
&
\code{size\_hint, size\_hint\_x,} and \code{size\_hint\_y} can always be set as \code{None} in favor
of \code{size, width,} and \code{height}.\\\hline

\end{tabular}


%    \clearpage% Flush page
%    \thispagestyle{empty}% empty page style (?)
%        \centering % Center table
%        \input{tab-conj-regular-ex-brief}
%        \twotabvspace
%        \input{tab-conj-regular-ex-brief}

    \clearpage% Flush page
}
\end{document}

